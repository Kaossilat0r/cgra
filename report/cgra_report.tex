\documentclass[oneside,11pt,accentcolor=tud2b, nochapname]{tudexercise}
\usepackage[utf8]{inputenc}
\usepackage[ngerman]{babel}

\usepackage{ngerman} % umlaute...
\usepackage{german} % umlaute...

\usepackage{hyperref}
\usepackage{color}
\usepackage{listings}
\usepackage{graphicx}

% Hurenkinder und Schusterjungen verhindern
\clubpenalty10000
\widowpenalty1000
\displaywidowpenalty=10000

% remove space between items 
\usepackage{enumitem}
	\setenumerate{noitemsep}
	\setitemize{noitemsep}
	\setdescription{noitemsep}

\title{Rechnersysteme 2}
\subtitle{CGRA Versuch\\
Wintersemester 2013/2014}

\subsubtitle{\\
Sebastian Fahnenschreiber (\href{mailto:sebastian.fahnenschreiber@stud.tu-darmstadt.de}{sebastian.fahnenschreiber@stud.tu-darmstadt.de})\\
Roman Neß (\href{mailto:roman.ness@stud.tu-darmstadt.de}{roman.ness@stud.tu-darmstadt.de})}

\selectlanguage{german}


\geometry{left=2.25cm,right=1.66cm,top=2.cm,bottom=2.25cm,includeheadfoot,heightrounded}
\begin{document}

\maketitle

\section*{Herangehensweise}
Wir handelten frei nach dem Motto: "`Make It Work, Make It Right, Make It Fast"':

\begin{itemize}
	\item erste Version in "`intuitivem"' C-Code
	\item Test der korrekten Funktionsweise bei nativer Kompilierung (mittels printf)
	\item Benutzung der Toolchain ohne Auslagerung von Berechnungen auf CGRA testen
	\item \textbf{dresc\_params\_v3} auf verwendete Schleifen anpassen und unoptimierte Performance messen
	\item Optimierungs-Zyklus beginnen (Ändern, Compilieren, Auswerten, Repeat)
\end{itemize}

\paragraph{Algorithmus}
Das Problem lässt sich in vier Teilschritte aufteilen, die im Programm in vier For-Schleifen bearbeitet werden.
Innerhalb einer Ausführung des Schleifenbodys wird jeweils eine Teilmatrix mit acht Elementen bearbeitet.
\begin{itemize}
	\item Hinrichtung für Zeilen. (8x1 Teilmatrizen)
	\item Hinrichtung für Spalten (1x8 Teilmatrizen)
	\item \textcolor{gray}{(Ausgabe des Zwischenergebnis)}
	\item Rückrichtung für Spalten.
	\item Rückrichtung für Zeilen.
	\item \textcolor{gray}{(Ausgabe des Endergebnis)}
\end{itemize}

\section*{Optimierungen Performance-Max}
Wir entschieden uns zunächst den Code für das Ziel "`performance-max"' hin zu optimieren.
Hauptgrund hierfür war das klar verständliche Kriterium von möglichst wenigen Taktzyklen um jeden Preis.\\
\\
Wir verwenden die "`6x6"' Instanz, weil davon auszugehen ist, dass mehr FUs mehr parallele Arbeit in kürzerer Zeit bewältigen können.
Da ein Leerlauf einzelner FUs das Ergebnis nicht negativ beeinflussen kann, sahen wir keinen Grund die kleineren Instanzen zu testen.\\
\\
Wir gehen davon aus, dass sich die Performance noch weiter steigern lassen könnte, wenn ein noch größeres Array verwendet wird.
Besonders müsste das für jede Schleifen gelten, in denen der Compiler wegen mangelnder Ressourcen ein erhöhtes Initiation Interval erzwingt.
Allerdings konnten wir nicht herausfinden wie sich ein eigenes Array automatisch erzeugen lässt und ein händisches Erstellen schien den Rahmen dieses Versuchs klar zu sprengen.\\
\\
Konkret wurden die folgenden Optimierungstechniken angewandt:
\begin{description}	
	\item[Kein Loop Nesting] \hfill \\
	Alle geschachtelten Schleifen wurden aufgelöst um den Kontrollfluss auf dem CGRA zu vereinfachen und Overhead durch die Schleifen zu verhindern.\\
	
	\item[Minimierung der Speicherzugriffe] \hfill \\
	Da Speicherzugriffe die teuersten Operation sind, wurde die Anzahl der Speicherzugriffe auf das absolute Minimum reduziert.
	Das heißt, jedes Array-Element wird in einer Transformation genau einmal gelesen und genau einmal geschrieben.
	Dies lässt sich auch in der Ausgabe auf der Konsole nachvollziehen (576 Elemente * 4 Transformationen = 2304 [+1 Zugriff für Basisadresse]).\\
	
	\item[Vorausberechnung von Konstanten] \hfill \\
	Alle Konstanten die schon zur Compile-Zeit vorberechnet werden können, werden als Makros definiert. 
	Darunter fallen z.B. Startindizes der Matrix oder Offsets zwischen Elementen.\\
	
	\item[Method Inlining] \hfill \\
	Um Methodenaufrufe auf dem CGRA zu verhindern und den Code gleichzeitig lesbar zu halten, werden Unterfunktionen in optimierten Schleifen als inline definiert.
	Da der Compiler bei Tests den "`inline"' Modifier mitunter ignorierte, erzwingen wir  Inlining per Compiler Attribut \texttt{\_\_attribute\_\_((always\_inline))}.\\
	
	\item[Intrinsics] \hfill \\
	Zur Berechnung des Durchschnitts verwenden wir ein eingebautes Intrinsic der Architektur (\texttt{intr12\_gp\_\_avg}).\\
	
	\item[Verwendete Parameter] \hfill \\
	\begin{itemize}
		\item \textbf{II} Kleinstmöglicher Wert (Wird durch RESMII nach unten hin begrenzt, Ziel ist es diesen Wert zu erreichen)
		\item \textbf{Random Seed} Standardwert: 2006, wenn das Mapping mit dem Standartwert nicht mit dem minimalen II funktioniert, solange am Seed drehen, bis es klappt
		\item \textbf{Relax Factor} Kleinstmöglicher wert (= 1.3) verwenden.
			Wenn das Mapping automatisch einen größeren Wert verwendet, diesen eintragen, damit es beim nächsten mal nicht alle kleineren Faktoren probiert werden müssen und es schneller geht.
	\end{itemize}
	
	\item[Optimierung der Ausgabefunktion] \hfill \\
	Da zunächst Kommuniziert wurde, dass die Ausgabe der Matrix Teil der Performance Messung sein muss, optimierten wir auch diese.
	Es zeigte sich, dass bei der Ausgabefunktion im Vergleich zur Berechnung ein viel größeres Optimierungspotential besteht.\\
	Nach der Optimierung mittels massivem loop-unrolling benötigte die Ausgabe der 24x24 Matrix bei unserer Lösung etwa 800 Taktzyklen.
	Aufgrund der Komplexität des entstanden Codes fielen wir wieder auf eine unoptimierte Version zurück, als bekannt wurde, dass die Ausgabe nicht Teil der Messung ist.
	
\end{description}


\section*{Herausforderungen}

Die Benutzung der Toolchain war nicht frei von Hindernissen.
Zu Beginn des Versuchs war die bereitgestellte VM nicht in einem Zustand, in dem wir sie für unseren Workflow gebrauchen konnten.
So musste z.B. ssh eingerichtet werden und die Befehle zur Verwendung der Toolchain aus einer Geany Konfigurationsdatei extrahiert werden.\\
\\
Weiterhin hat das Shellscript zur Ausgabe bei uns zu Beginn nicht vollständig funktioniert. 
Wir haben es deswegen aus dem /usr Verzeichnis in den Workspace kopiert und repariert.
Die Berechnung der Ergebnisse blieb dabei unangetastet.\\
\\
Auch der Komfort während des Entwickelns war nicht wirklich gut.
Insbesondere fielen die Geschwindigkeit der Toolchain und die Fehleranfälligkeit und häufig notwendige Anpassungen in der Datei "`dresc\_params\_v3"' negativ auf.
Ebenso empfanden wir es als mühsam, dass es vom Finden eines passenden zufälligen Seeds abhängt, ob das Problem mit den angestrebten Parametern auf die Architektur gemappt werden kann.\\
\\
Alles in allem war das Projekt sehr gut machbar, jedoch konnte das Optimierungspotential nicht vollständig ausgenutzt werden, da z.B. die XML-Konfiguration der Instanzen undurchsichtig war.
Ein Tool zur Erstellung von eigenen Instanzen (oder zumindest die Visualisierung einer bestehenden Instanz) hätte uns hier sehr geholfen.
Leider konnte das Manual aufgrund des Aufbaus nicht als Nachschlagewerk helfen.
Abschließend ist zu sagen, dass wir wohl an einem Punkt angelangt sind, an dem weitere Optimierungen zwar möglich aber vom Zeitaufwand her nicht mehr vertretbar sind.

\lstset{
language=C,
basicstyle=\tiny\ttfamily,
numbers=left,
numberstyle=\tiny,
frame=tb,
keepspaces=true, 
columns=fullflexible,
showstringspaces=false,
commentstyle=\itshape\sffamily\color{gray!40!black},
}

\begin{lstlisting}[caption={Endausgabe für das Ziel Performance-Max (Instanz 6x6) },captionpos=b, float, label={lst:performance}]
**************
** statistics **
**************
Instr          	             Num	  Clk/Instr	      Energy/Clk	    Total energy	
------------------------------------------------------------------------------------------------
mov            	            2903	          1	    0.9699235897	 2815.6881808991	
add            	            1347	          1	             1.0	          1347.0	
add_u          	             214	          1	             1.0	           214.0	
sub            	            2254	          1	    0.9995122744	 2252.9006664976	
setlo          	               9	          1	    0.9699235897	    8.7293123073	
sethi          	               8	          1	    0.9699235897	    7.7593887176	
cga            	               4	          3	             1.5	            18.0	
intr12_gp existiert nicht!
intr12_gp      	            1121	          1	             1.0	          1121.0	
ld_i           	            2305	          4	    0.9705738904	 8948.6912694880	
pred_gt        	              80	          1	    0.9741505446	   77.9320435680	
pred_lt        	             135	          1	    0.9731750935	  131.3786376225	
st_i           	            2305	          1	    0.9721996423	 2240.9201755015	
rts            	               1	          3	             1.5	             4.5	
nop            	            5481	          1	             0.3	          1644.3	



**************
** overview **
**************
Used instance : 6x6
Total clocks on cga : 644
Total clocks  : 697
Total nop-clk : 5481 
Total energy  : 20832.7996746016
********************************
********************************
WARNING: No energy data for instruction: intr12_gp. Ignoring
+------------+--------------+---------------------+
| Ins        | Clk/ Energy/ |        Total        |
|            |  Ins     Clk |     Num      Energy |
+------------+--------------+---------------------+
| add        |    1  1.0000 |    1347      1347.0 |
| add_u      |    1  1.0000 |     214       214.0 |
| cga        |    3  1.5000 |       4        18.0 |
| ld_i       |    4  0.9706 |    2305      8948.7 |
| mov        |    1  0.9699 |    2903      2815.7 |
| nop        |    1  0.3000 |    6602      1980.6 |
| pred_gt    |    1  0.9742 |      80        77.9 |
| pred_lt    |    1  0.9732 |     135       131.4 |
| rts        |    3  1.5000 |       1         4.5 |
| sethi      |    1  0.9699 |       8         7.8 |
| setlo      |    1  0.9699 |       9         8.7 |
| st_i       |    1  0.9722 |    2305      2240.9 |
| sub        |    1  0.9995 |    2254      2252.9 |
+------------+--------------+---------------------+
+---------------+---------------------+
| Section       |        Total        |
+---------------+---------------------+
| FU setup      |                 6x6 |
| Cycles        |                 697 |
| CGRA cycles   |                 644 |
| NOPs          |                6602 |
| NOP %         |               26.3% |
| Reads         |                2305 |
| Writes        |                2305 |
| Total Energy  |               20048 |
| CGRA Energy   |                     |
+---------------+---------------------+
\end{lstlisting}

\begin{lstlisting}[caption={Endausgabe für das Ziel Energy-Min (Instanz 6x6++) },captionpos=b, float, label={lst:energy}]
**************
** statistics **
**************
Instr          	             Num	  Clk/Instr	      Energy/Clk	    Total energy	
------------------------------------------------------------------------------------------------
mov            	            2421	          1	    0.9699235897	 2348.1850106637	
add            	            1344	          1	             1.0	          1344.0	
add_u          	             214	          1	             1.0	           214.0	
sub            	            2255	          1	    0.9995122744	 2253.9001787720	
setlo          	               9	          1	    0.9699235897	    8.7293123073	
sethi          	               8	          1	    0.9699235897	    7.7593887176	
cga            	               4	          3	             1.5	            18.0	
intr12_gp existiert nicht!
intr12_gp      	            1122	          1	             1.0	          1122.0	
ld_i           	            2305	          4	    0.9705738904	 8948.6912694880	
pred_gt        	             160	          1	    0.9741505446	  155.8640871360	
pred_lt        	              54	          1	    0.9731750935	   52.5514550490	
st_i           	            2305	          1	    0.9721996423	 2240.9201755015	
rts            	               1	          3	             1.5	             4.5	
nop            	            5965	          1	             0.3	          1789.5	



**************
** overview **
**************
Used instance : 6x6
Total clocks on cga : 644
Total clocks  : 697
Total nop-clk : 5965 
Total energy  : 20508.6008776351
********************************
********************************
WARNING: No energy data for instruction: intr12_gp. Ignoring
+------------+--------------+---------------------+
| Ins        | Clk/ Energy/ |        Total        |
|            |  Ins     Clk |     Num      Energy |
+------------+--------------+---------------------+
| add        |    1  1.0000 |    1344      1344.0 |
| add_u      |    1  1.0000 |     214       214.0 |
| cga        |    3  1.5000 |       4        18.0 |
| ld_i       |    4  0.9706 |    2305      8948.7 |
| mov        |    1  0.9699 |    2421      2348.2 |
| nop        |    1  0.3000 |    7087      2126.1 |
| pred_gt    |    1  0.9742 |     160       155.9 |
| pred_lt    |    1  0.9732 |      54        52.6 |
| rts        |    3  1.5000 |       1         4.5 |
| sethi      |    1  0.9699 |       8         7.8 |
| setlo      |    1  0.9699 |       9         8.7 |
| st_i       |    1  0.9722 |    2305      2240.9 |
| sub        |    1  0.9995 |    2255      2253.9 |
+------------+--------------+---------------------+
+---------------+---------------------+
| Section       |        Total        |
+---------------+---------------------+
| FU setup      |                 6x6 |
| Cycles        |                 697 |
| CGRA cycles   |                 644 |
| NOPs          |                7087 |
| NOP %         |               28.2% |
| Reads         |                2305 |
| Writes        |                2305 |
| Total Energy  |               19723 |
| CGRA Energy   |                     |
+---------------+---------------------+
\end{lstlisting}

\end{document}


